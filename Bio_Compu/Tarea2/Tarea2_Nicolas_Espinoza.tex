\documentclass[a4paper, 12pt, notitlepage]{article}
\usepackage[margin=0.5in]{geometry}
\usepackage[english]{babel}
\usepackage[utf8x]{inputenc}
\usepackage{fancyhdr}
\usepackage{amsmath}
\usepackage{amssymb}
\usepackage{gensymb}
\usepackage[makeroom]{cancel}
\usepackage{tikz}
\usepackage{float}
\usepackage{pdfpages}
\usepackage{graphicx}
\usepackage{tabularx}
\usepackage{hyperref}

\title{Computational Biology - 2$^\text{nd}$ Assignment}
\author{Nicolás Espinoza Muñoz}
\date{October 30, 2020}

\newcommand\numberthis{\addtocounter{equation}{1}\tag{\theequation}}
\pagenumbering{gobble}

\begin{document}
\maketitle
\subsubsection*{Introduction}
\section*{Electrostatic theoretical outline}
The electrostatic interactions between particles are calculated for all particles in a system. Since a single reference particle is affected by all other components of the system, and one has to cycle through all of said components as reference to obtain the electrostatic potential of the ensemble, this calculation is of order $\mathcal{O}(N^2)$. However, there are acceleration methods that help achieve the results needed much faster, but at the expense of a bit of precision. For starters we will present the basic electrostatic energy formula, which is the basic way of calculating interactions, and the one that grows as $\mathcal{O}(N^2)$. In the same section, the Ewald Summation and PME methods will be introduced. Finally, a comparison between methods will be shown so as to appreciate the differences between them.
\subsection*{Direct Calculation}
For a system like the one shown in Figure 1, interactions are calculated in pairs, then summed over all possible pairs for a specific $i$ particle interacting with all $j$, and summed again for all $i$ particles in the ensemble.
\begin{equation}
E = \frac{1}{2}\sum_{i = 1}^{N}\sum_{\substack{j=1\\j\neq i}}^{N}\frac{1}{4\pi\varepsilon}\frac{q_i\cdot q_j}{|\mathbf{r}_i - \mathbf{r}_j|}\quad \implies \quad E = \frac{1}{2}\sum_{i=1}^{N}q_i\cdot\phi(\mathbf{r}_i)\label{eq:eq1}
\end{equation}
For a continuous charge distribution one has to integrate over the volume of interest and afterwards proceed with the double summations, but the procedure stays the same.\\\\
It will become useful to introduce the Poisson equation of electrostatics, after which one can calculate the potential in terms of the charge density
\begin{equation}
	-\nabla^2\phi = \frac{\rho}{\varepsilon}\label{eq:eq2}
\end{equation}
There's not much more to elaborate on this, except for the fact that it is easy to see how the computation time escalates with the square of the number of particles N. With periodic boundary conditions this is obviously impossible to calculate, because it would mean an infinite double summation. So instead we proceed with the Ewald Summation method.
\subsection*{Ewald Summation}
The Ewald Summation takes a slowly converging potential, and splits it into two parts: a short range one and a long range one. For this, one starts by actually splitting the charge density $\rho$, and adding a zero via a Gaussian distribution
\begin{equation}
\rho_i = \rho_i^S + \rho_i^L = q_i\delta(\mathbf{r} - \mathbf{r}_i) - q_i G_\sigma(\mathbf{r} - \mathbf{r}_i) + q_i G_\sigma(\mathbf{r} - \mathbf{r}_i)\label{eq:eq3}
\end{equation}
where
\begin{align*}
&\rho_i^S = q_i\delta(\mathbf{r} - \mathbf{r}_i) - q_i G_\sigma (\mathbf{r} - \mathbf{r}_i)\\
&\rho_i^L = q_i G_\sigma (\mathbf{r} - \mathbf{r}_i)
\end{align*}
So using Equation (\ref{eq:eq3}) into Equation (\ref{eq:eq2}) to obtain a split potential ($\phi^S$ and $\phi^L$) we get
\begin{gather*}
	\phi_i^S = \sum_{\substack{j=1\\j\neq i}}^{N}\frac{q_j}{4\pi\varepsilon}\int\frac{\delta(\mathbf{r}_i - \mathbf{r}') - G_\sigma (\mathbf{r}_i - \mathbf{r}')}{|\mathbf{r}_i - \mathbf{r}'|}\, dV'\\
	\phi_i^L = \sum_{\substack{j = 1\\j\neq i}}^{N}\frac{q_j}{4\pi\varepsilon} \int \frac{G_\sigma(\mathbf{r}_i - \mathbf{r}')}{|\mathbf{r}_i - \mathbf{r}'|}\, d V' \numberthis \label{eq:eq4}
\end{gather*}
The first integral of Equation (\ref{eq:eq4}) corresponds to the $erfc(x) = 1 - erf(x)$, the error function complement, and the second integral is the error function itself. The logic behind this procedure is that the short range potential is singular, but converges quickly. The long range potential on the other hand is not singular, and accounts for the slowly converging nature of the actual physical potential, and the sum of both returns the original $\phi$.\\\\
So since the short range potential converges quickly, we can calculate it directly, which leaves the long range potential as the only actual dilemma. For this, one can apply a Fourier Transform to obtain the potential in reciprocal space - that is, in terms of frequencies. By transforming the charge density to Fourier Space, we can easily calculate the long range potential
\begin{gather*}
	\mathcal{F} (\rho) = \hat{\rho} = \int\rho^L\, e^{-i\mathbf{k}\cdot\mathbf{r}}\, dV\\
	\mathcal{F} (\phi) = \hat{\phi} = \int\phi^L \, e^{-i\mathbf{k}\cdot\mathbf{r}}\, dV
\end{gather*}
So to calculate the long range potential in real space, one calculates it in reciprocal space first by applying the equivalent form of Equation (\ref{eq:eq2})
\begin{equation}
	\hat{\phi}^L = \frac{\hat{\rho}^L}{\varepsilon |\mathbf{k}|^2}\label{eq:eq5}
\end{equation}
And then use the inverse Fourier Transform to recover the real space version of the result. The factor $1/|\mathbf{k}|^2$ in Equation (\ref{eq:eq5}) comes from the Laplace operator applied to the potential, and $\mathbf{k}$ is  the reciprocal vector. The idea behind working in Fourier Space is that the summation for $\phi^L$ is actually short range in reciprocal space, so one can calculate it with few terms in the summation to a decent precision. The resulting expression for the real form of the long range potential is
\begin{equation}
	\phi^L = \frac{1}{V}\sum_{k=1}\hat{\phi}^L e^{i\mathbf{k}\cdot\mathbf{r}}= \frac{1}{V\varepsilon}\sum_{k = 1}\sum_{j = 1}^N \frac{q_j}{|\mathbf{k}|^2}e^{i\mathbf{k}\cdot(\mathbf{r} - \mathbf{r}_j)}e^{\beta}\label{eq:eq6}
\end{equation}
Where $\beta = -\sigma^2|\mathbf{k}|^2/2$, and $\sigma$ is the standard deviation of the error function.
Finally, we sum both long and short range potentials to recover the original, physical potential. This method, as is, converges with $\mathcal{O}(N^{3/2})$, but it can be accelerated by means of the Fast Fourier Transform instead of a regular one for calculating the potential in reciprocal space. This makes the method a lot faster for big $N$, and the calculation time escalates with order $\mathcal{O}(N\log N)$; for this we have to explain the PME method, which makes use of the FFT acceleration algorithm.
\subsection{Particle-Mesh Ewald}
The PME method bases its application in the fact that, usually, one works with regular meshes rather on the specific particles, and for Periodic Boundary Conditions, the number of mesh elements will more likely than not be smaller than the number of particles interacting. The method consists of distributing the charge of a particle onto the mesh. The potential is then calculated in Fourier Space on the mesh points, and brought back to real space by means of the inverse Fast Fourier Transform. Figure \ref{fig:fig2} shows the idea in a 1-D setting, using a Gaussian, but the distribution itself may vary. As stated in the previous section, PME is of order $\mathcal{O}(N\log N)$.
\begin{figure}
	\centering
	\input{./Figures/PME.eps_tex}
	\caption{Distribution example for a one-dimensional test charge.}
\end{figure}
\section*{Comparison between calculation methods}
\end{document}