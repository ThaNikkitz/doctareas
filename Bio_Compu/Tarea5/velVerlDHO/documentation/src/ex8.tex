\documentclass[a4paper,12pt]{article}
%\batchmode
\usepackage[dvips]{color}
\usepackage{lscape}
\usepackage{amsmath}
\usepackage{fancyhdr}
\usepackage{graphicx}
\usepackage{lscape}
\usepackage{amsfonts}
\usepackage{units} % for nicefrac
\voffset-1.5cm
\textwidth15.5cm
\textheight22.5cm
\hoffset-1.5cm

\setlength{\headheight}{60pt}

\renewcommand{\baselinestretch}{1.2}

\newcommand{\abs}[1]{\ensuremath{\left\vert\left\vert#1\right\vert\right\vert}}

\pagestyle{fancy}

\renewcommand{\headrulewidth}{0.4pt} %
\renewcommand{\footrulewidth}{0pt} %
\fancyhead[L]{
              Computational Biology Course
              } %
\fancyfoot[C]{\thepage} %

\newcommand{\deadline}[1]{{\bf Deadline:} #1\vspace{0.5cm}} %


\makeatletter
\def\maketitle{%
  \begin{center}
    {\Large \bf Exercise Sheet \@title\par}%
    \vskip 0.5cm
    {\normalfont \@date\par}%
  \end{center}%
    \vskip 1.0cm
  }
\makeatother






\begin{document}
% EDIT EXERCISE NUMBER AND EXERCISE DATE
\title{{Velocity Verlet MD}}
\date{}
\maketitle


% INSERT QUESTIONS HERE
%
\renewcommand{\labelenumi}{\arabic{enumi}.}
%
%
%
\begin{enumerate}
%\item Consider a pendulum of length $l$ with mass $m$, as shown in Figure 1.
%A gravitational field of uniform acceleration $g$ is acting on the mass.
%\begin{figure}[h!]
%\vspace{1cm}
%\centering
%\includegraphics[width=0.4\textwidth]{pic}
%\caption{A simple pendulum of length $l$ and mass $m$. $\theta$ denotes the angle between the pendulum and the vertical.}
%\end{figure}
%\begin{itemize}
%\item Verify that the potential energy of the pendulum is given by $E_{\mathit pot} =   m g l ( 1 - \cos \theta)$, where $\theta$ denotes the angle between the pendulum and the vertical as shown in Figure 1. 
%\item Based on the  generalized coordinate $\theta$, write down the
%Lagrangian of the system and the equation of motion using the Lagrangian formalism.
%\item Show that the
%conjugate momentum of $\theta$ is $p_{\theta}=ml^2\dot{\theta}$, 
%write down the
%Hamiltonian of the system and the equations of motion using the Hamiltonian formalism.
%\end{itemize}
\item A C-program \texttt{velocityVerletSho.c} is provided which integrates
the equation of motion for the simple harmonic oscillator according to the Velocity-Verlet algorithm. For simplicity, the mass and force constant are defined
as  \texttt{MASS=1} and \texttt{K=1}, respectively. The initial conditions are defined as $\texttt{X0=1}$ (coordinate)
and \texttt{V0=0} (velocity). 
The number of integration steps is \texttt{NSTEPS}  and the time step size is given by \texttt{deltat = Tper / TFAC}, where \texttt{Tper} is the period of oscillation and \texttt{TFAC} is the number of time steps per period of oscillation.
The program can be compiled with\\
\noindent \texttt{gcc -o velocityVerletSho program/velocityVerletSho.c}\\
\noindent and run with\\
\noindent \texttt{./velocityVerletSho} \\
\noindent For each time step, it prints the time in units of period, coordinate, momentum, total energy 
\begin{equation}
E_{\mathit{tot}} = \frac{1}{2} m v^2 + \frac{1}{2}  k x^2
\end{equation}
 %and the energy of the shadow Hamiltonian,
%\begin{equation}
%\tilde{E}_{\mathit{tot}} = \frac{1}{2} m v^2 \left[1-(\omega \Delta t/2)^2\right]^{-1} + %\frac{1}{2}  k x^2 \quad .
%\end{equation}
The output can be written to a file \texttt{out.dat} with \\
\noindent  \texttt{./velocityVerletSho} $>$ \texttt{out.dat} \\
\noindent That is, in this file, there are five columns which report
the time in units of period, coordinate, momentum, total energy and the energy of the shadow Hamiltonian, respectively.
\begin{itemize}
\item Look at the program and verify the equation of motion of the simple harmonic oscillator, as well as the equations of the Velocity-Verlet algorithm.
\item Edit, compile and run the program for variables \texttt{TFAC=10}, \texttt{50} and \texttt{100} and write corresponding output data to files
\texttt{out10.dat},
\texttt{out50.dat} and
\texttt{out100.dat}, respectively.
\item Make phase space plots for the three above runs, {\em i.e.} plot the momentum {\em vs.} the coordinate. Interpret.
\item Plot the energy {\em vs.} time. Interpret.
%\item Plot the energy of the shadow Hamiltonian {\em vs.} time. It is conserved exactly by the Velocity-Verlet algorithm. Thus, is the Velocity-Verlet algorithm symplectic?
\end{itemize}
\end{enumerate}
%
%
%
\end{document}
